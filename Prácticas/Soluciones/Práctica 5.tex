\documentclass{article}
\usepackage{ifthen}
\usepackage{amssymb}
\usepackage{multicol}
\usepackage{graphicx}
\usepackage[absolute]{textpos}
\usepackage{amsmath, amscd, amssymb, amsthm, latexsym}
% \usepackage[noload]{qtree}
%\usepackage{xspace,rotating,calligra,dsfont,ifthen}
\usepackage{xspace,rotating,dsfont,ifthen}
\usepackage[spanish,activeacute]{babel}
\usepackage[utf8]{inputenc}
\usepackage{pgfpages}
\usepackage{pgf,pgfarrows,pgfnodes,pgfautomata,pgfheaps,xspace,dsfont}
\usepackage{listings}
\usepackage{multicol}
\usepackage{todonotes}
\usepackage{url}
\usepackage{float}
\usepackage{framed,mdframed}
\usepackage{cancel}

\usepackage[strict]{changepage}


\makeatletter


\newcommand\hfrac[2]{\genfrac{}{}{0pt}{}{#1}{#2}} %\hfrac{}{} es un \frac sin la linea del medio

\newcommand\Wider[2][3em]{% \Wider[3em]{} reduce los m\'argenes
\makebox[\linewidth][c]{%
  \begin{minipage}{\dimexpr\textwidth+#1\relax}
  \raggedright#2
  \end{minipage}%
  }%
}


\@ifclassloaded{beamer}{%
  \newcommand{\tocarEspacios}{%
    \addtolength{\leftskip}{4em}%
    \addtolength{\parindent}{-3em}%
  }%
}
{%
  \usepackage[top=1cm,bottom=2cm,left=1cm,right=1cm]{geometry}%
  \usepackage{color}%
  \newcommand{\tocarEspacios}{%
    \addtolength{\leftskip}{3em}%
    \setlength{\parindent}{0em}%
  }%
}

\newcommand{\encabezadoDeProc}[4]{%
  % Ponemos la palabrita problema en tt
%  \noindent%
  {\normalfont\bfseries\ttfamily proc}%
  % Ponemos el nombre del problema
  \ %
  {\normalfont\ttfamily #2}%
  \
  % Ponemos los parametros
  (#3)%
  \ifthenelse{\equal{#4}{}}{}{%
  \ =\ %
  % Ponemos el nombre del resultado
  {\normalfont\ttfamily #1}%
  % Por ultimo, va el tipo del resultado
  \ : #4}
}

\newcommand{\encabezadoDeTipo}[2]{%
  % Ponemos la palabrita tipo en tt
  {\normalfont\bfseries\ttfamily tipo}%
  % Ponemos el nombre del tipo
  \ %
  {\normalfont\ttfamily #2}%
  \ifthenelse{\equal{#1}{}}{}{$\langle$#1$\rangle$}
}

% Primero definiciones de cosas al estilo title, author, date

\def\materia#1{\gdef\@materia{#1}}
\def\@materia{No especifi\'o la materia}
\def\lamateria{\@materia}

\def\cuatrimestre#1{\gdef\@cuatrimestre{#1}}
\def\@cuatrimestre{No especifi\'o el cuatrimestre}
\def\elcuatrimestre{\@cuatrimestre}

\def\anio#1{\gdef\@anio{#1}}
\def\@anio{No especifi\'o el anio}
\def\elanio{\@anio}

\def\fecha#1{\gdef\@fecha{#1}}
\def\@fecha{\today}
\def\lafecha{\@fecha}

\def\nombre#1{\gdef\@nombre{#1}}
\def\@nombre{No especific'o el nombre}
\def\elnombre{\@nombre}

\def\practicas#1{\gdef\@practica{#1}}
\def\@practica{No especifi\'o el n\'umero de pr\'actica}
\def\lapractica{\@practica}


% Esta macro convierte el numero de cuatrimestre a palabras
\newcommand{\cuatrimestreLindo}{
  \ifthenelse{\equal{\elcuatrimestre}{1}}
  {Primer cuatrimestre}
  {\ifthenelse{\equal{\elcuatrimestre}{2}}
  {Segundo cuatrimestre}
  {Verano}}
}


\newcommand{\depto}{{UBA -- Facultad de Ciencias Exactas y Naturales --
      Departamento de Computaci\'on}}

\newcommand{\titulopractica}{
  \centerline{\depto}
  \vspace{1ex}
  \centerline{{\Large\lamateria}}
  \vspace{0.5ex}
  \centerline{\cuatrimestreLindo de \elanio}
  \vspace{2ex}
  \centerline{{\huge Pr\'actica \lapractica -- \elnombre}}
  \vspace{5ex}
  \arreglarincisos
  \newcounter{ejercicio}
  \newenvironment{ejercicio}{\stepcounter{ejercicio}\textbf{Ejercicio
      \theejercicio}%
    \renewcommand\@currentlabel{\theejercicio}%
  }{\vspace{0.2cm}}
}


\newcommand{\titulotp}{
  \centerline{\depto}
  \vspace{1ex}
  \centerline{{\Large\lamateria}}
  \vspace{0.5ex}
  \centerline{\cuatrimestreLindo de \elanio}
  \vspace{0.5ex}
  \centerline{\lafecha}
  \vspace{2ex}
  \centerline{{\huge\elnombre}}
  \vspace{5ex}
}


%practicas
\newcommand{\practica}[2]{%
    \title{Pr\'actica #1 \\ #2}
    \author{Algoritmos y Estructuras de Datos I}
    \date{Segundo Cuatrimestre 2019}

    \maketitlepractica{#1}{#2}
}

\newcommand \maketitlepractica[2] {%
\begin{center}
\begin{tabular}{r cr}
 \begin{tabular}{c}
{\large\bf\textsf{\ Algoritmos y Estructuras de Datos I\ }}\\
Primer Cuatrimestre 2021\\
\title{\normalsize Gu\'ia Pr\'actica #1 \\ \textbf{#2}}\\
\@title
\end{tabular} &
\begin{tabular}{@{} p{1.6cm} @{}}
\includegraphics[width=1.6cm]{logodpt.jpg}
\end{tabular} &
\begin{tabular}{l @{}}
 \emph{Departamento de Computaci\'on} \\
 \emph{Facultad de Ciencias Exactas y Naturales} \\
 \emph{Universidad de Buenos Aires} \\
\end{tabular}
\end{tabular}
\end{center}

\bigskip
}


% Símbolos varios

\newcommand{\nat}{\ensuremath{\mathds{N}}}
\newcommand{\ent}{\ensuremath{\mathds{Z}}}
\newcommand{\float}{\ensuremath{\mathds{R}}}
\newcommand{\bool}{\ensuremath{\mathsf{Bool}}}
\newcommand{\True}{\ensuremath{\mathrm{true}}}
\newcommand{\False}{\ensuremath{\mathrm{false}}}
\newcommand{\Then}{\ensuremath{\rightarrow}}
\newcommand{\Iff}{\ensuremath{\leftrightarrow}}
\newcommand{\implica}{\ensuremath{\longrightarrow}}
\newcommand{\IfThenElse}[3]{\ensuremath{\mathsf{if}\ #1\ \mathsf{then}\ #2\ \mathsf{else}\ #3\ \mathsf{fi}}}
\newcommand{\In}{\textsf{in }}
\newcommand{\Out}{\textsf{out }}
\newcommand{\Inout}{\textsf{inout }}
\newcommand{\yLuego}{\land _L}
\newcommand{\oLuego}{\lor _L}
\newcommand{\implicaLuego}{\implica _L}
\newcommand{\cuantificador}[5]{%
	\ensuremath{(#2 #3: #4)\ (%
		\ifthenelse{\equal{#1}{unalinea}}{
			#5
		}{
			$ % exiting math mode
			\begin{adjustwidth}{+2em}{}
			$#5$%
			\end{adjustwidth}%
			$ % entering math mode
		}
	)}
}

\newcommand{\existe}[4][]{%
	\cuantificador{#1}{\exists}{#2}{#3}{#4}
}
\newcommand{\paraTodo}[4][]{%
	\cuantificador{#1}{\forall}{#2}{#3}{#4}
}

% Símbolo para marcar los ejercicios importantes (estrellita)
\newcommand\importante{\raisebox{0.5pt}{\ensuremath{\bigstar}}}


\newcommand{\rango}[2]{[#1\twodots#2]}
\newcommand{\comp}[2]{[\,#1\,|\,#2\,]}

\newcommand{\rangoac}[2]{(#1\twodots#2]}
\newcommand{\rangoca}[2]{[#1\twodots#2)}
\newcommand{\rangoaa}[2]{(#1\twodots#2)}

%ejercicios
\newtheorem{exercise}{Ejercicio}
\newenvironment{ejercicio}[1][]{\begin{exercise}#1\rm}{\end{exercise} \vspace{0.2cm}}
\newenvironment{items}{\begin{enumerate}[a)]}{\end{enumerate}}
\newenvironment{subitems}{\begin{enumerate}[i)]}{\end{enumerate}}
\newcommand{\sugerencia}[1]{\noindent \textbf{Sugerencia:} #1}

\lstnewenvironment{code}{
    \lstset{% general command to set parameter(s)
        language=C++, basicstyle=\small\ttfamily, keywordstyle=\slshape,
        emph=[1]{tipo,usa}, emphstyle={[1]\sffamily\bfseries},
        morekeywords={tint,forn,forsn},
        basewidth={0.47em,0.40em},
        columns=fixed, fontadjust, resetmargins, xrightmargin=5pt, xleftmargin=15pt,
        flexiblecolumns=false, tabsize=2, breaklines, breakatwhitespace=false, extendedchars=true,
        numbers=left, numberstyle=\tiny, stepnumber=1, numbersep=9pt,
        frame=l, framesep=3pt,
    }
   \csname lst@SetFirstLabel\endcsname}
  {\csname lst@SaveFirstLabel\endcsname}


%tipos basicos
\newcommand{\rea}{\ensuremath{\mathsf{Float}}}
\newcommand{\cha}{\ensuremath{\mathsf{Char}}}
\newcommand{\str}{\ensuremath{\mathsf{String}}}

\newcommand{\mcd}{\mathrm{mcd}}
\newcommand{\prm}[1]{\ensuremath{\mathsf{prm}(#1)}}
\newcommand{\sgd}[1]{\ensuremath{\mathsf{sgd}(#1)}}

\newcommand{\tuple}[2]{\ensuremath{#1 \times #2}}

%listas
\newcommand{\TLista}[1]{\ensuremath{seq \langle #1\rangle}}
\newcommand{\lvacia}{\ensuremath{[\ ]}}
\newcommand{\lv}{\ensuremath{[\ ]}}
\newcommand{\longitud}[1]{\ensuremath{|#1|}}
\newcommand{\cons}[1]{\ensuremath{\mathsf{addFirst}}(#1)}
\newcommand{\indice}[1]{\ensuremath{\mathsf{indice}}(#1)}
\newcommand{\conc}[1]{\ensuremath{\mathsf{concat}}(#1)}
\newcommand{\cab}[1]{\ensuremath{\mathsf{head}}(#1)}
\newcommand{\cola}[1]{\ensuremath{\mathsf{tail}}(#1)}
\newcommand{\sub}[1]{\ensuremath{\mathsf{subseq}}(#1)}
\newcommand{\en}[1]{\ensuremath{\mathsf{en}}(#1)}
\newcommand{\cuenta}[2]{\mathsf{cuenta}\ensuremath{(#1, #2)}}
\newcommand{\suma}[1]{\mathsf{suma}(#1)}
\newcommand{\twodots}{\ensuremath{\mathrm{..}}}
\newcommand{\masmas}{\ensuremath{++}}
\newcommand{\matriz}[1]{\TLista{\TLista{#1}}}

\newcommand{\seqchar}{\TLista{\cha}}


% Acumulador
\newcommand{\acum}[1]{\ensuremath{\mathsf{acum}}(#1)}
\newcommand{\acumselec}[3]{\ensuremath{\mathrm{acum}(#1 |  #2, #3)}}

% \selector{variable}{dominio}
\newcommand{\selector}[2]{#1~\ensuremath{\leftarrow}~#2}
\newcommand{\selec}{\ensuremath{\leftarrow}}

\newcommand{\pred}[3]{%
    {\normalfont\bfseries\ttfamily\noindent pred }%
    {\normalfont\ttfamily #1}%
    \ifthenelse{\equal{#2}{}}{}{\ (#2) }%
    \{%
    \begin{adjustwidth}{+2em}{}
      \ensuremath{#3}
    \end{adjustwidth}
    \}%
    {\normalfont\bfseries\,\par}%
}

\newenvironment{proc}[4][res]{%

  % El parametro 1 (opcional) es el nombre del resultado
  % El parametro 2 es el nombre del problema
  % El parametro 3 son los parametros
  % El parametro 4 es el tipo del resultado
  % Preambulo del ambiente problema
  % Tenemos que definir los comandos requiere, asegura, modifica y aux
  \newcommand{\pre}[2][]{%
    {\normalfont\bfseries\ttfamily Pre}%
    \ifthenelse{\equal{##1}{}}{}{\ {\normalfont\ttfamily ##1} :}\ %
    \{\ensuremath{##2}\}%
    {\normalfont\bfseries\,\par}%
  }
  \newcommand{\post}[2][]{%
    {\normalfont\bfseries\ttfamily Post}%
    \ifthenelse{\equal{##1}{}}{}{\ {\normalfont\ttfamily ##1} :}\
    \{\ensuremath{##2}\}%
    {\normalfont\bfseries\,\par}%
  }
  \renewcommand{\aux}[4]{%
    {\normalfont\bfseries\ttfamily aux\ }%
    {\normalfont\ttfamily ##1}%
    \ifthenelse{\equal{##2}{}}{}{\ (##2)}\ : ##3\, = \ensuremath{##4}%
    {\normalfont\bfseries\,;\par}%
  }
  \renewcommand{\pred}[3]{%
    {\normalfont\bfseries\ttfamily pred }%
    {\normalfont\ttfamily ##1}%
    \ifthenelse{\equal{##2}{}}{}{\ (##2) }%
    \{%
    \begin{adjustwidth}{+5em}{}
      \ensuremath{##3}
    \end{adjustwidth}
    \}%
    {\normalfont\bfseries\,\par}%
  }

  \newcommand{\res}{#1}
  \vspace{1ex}
  \noindent
  \encabezadoDeProc{#1}{#2}{#3}{#4}
  % Abrimos la llave
  \{\par%
  \tocarEspacios
}
% Ahora viene el cierre del ambiente problema
{
  % Cerramos la llave
  \noindent\}
  \vspace{1ex}
}


\newcommand{\aux}[4]{%
    {\normalfont\bfseries\ttfamily\noindent aux\ }%
    {\normalfont\ttfamily #1}%
    \ifthenelse{\equal{#2}{}}{}{\ (#2)}\ : #3\, = \ensuremath{#4}%
    {\normalfont\bfseries\,;\par}%
}


% \newcommand{\pre}[1]{\textsf{pre}\ensuremath{(#1)}}

\newcommand{\procnom}[1]{\textsf{#1}}
\newcommand{\procil}[3]{\textsf{proc #1}\ensuremath{(#2) = #3}}
\newcommand{\procilsinres}[2]{\textsf{proc #1}\ensuremath{(#2)}}
\newcommand{\preil}[2]{\textsf{Pre #1: }\ensuremath{#2}}
\newcommand{\postil}[2]{\textsf{Post #1: }\ensuremath{#2}}
\newcommand{\auxil}[2]{\textsf{fun }\ensuremath{#1 = #2}}
\newcommand{\auxilc}[4]{\textsf{fun }\ensuremath{#1( #2 ): #3 = #4}}
\newcommand{\auxnom}[1]{\textsf{fun }\ensuremath{#1}}
\newcommand{\auxpred}[3]{\textsf{pred }\ensuremath{#1( #2 ) \{ #3 \}}}

\newcommand{\comentario}[1]{{/*\ #1\ */}}

\newcommand{\nom}[1]{\ensuremath{\mathsf{#1}}}


% En las practicas/parciales usamos numeros arabigos para los ejercicios.
% Aca cambiamos los enumerate comunes para que usen letras y numeros
% romanos
\newcommand{\arreglarincisos}{%
  \renewcommand{\theenumi}{\alph{enumi}}
  \renewcommand{\theenumii}{\roman{enumii}}
  \renewcommand{\labelenumi}{\theenumi)}
  \renewcommand{\labelenumii}{\theenumii)}
}



%%%%%%%%%%%%%%%%%%%%%%%%%%%%%% PARCIAL %%%%%%%%%%%%%%%%%%%%%%%%
\let\@xa\expandafter
\newcommand{\tituloparcial}{\centerline{\depto -- \lamateria}
  \centerline{\elnombre -- \lafecha}%
  \setlength{\TPHorizModule}{10mm} % Fija las unidades de textpos
  \setlength{\TPVertModule}{\TPHorizModule} % Fija las unidades de
                                % textpos
  \arreglarincisos
  \newcounter{total}% Este contador va a guardar cuantos incisos hay
                    % en el parcial. Si un ejercicio no tiene incisos,
                    % cuenta como un inciso.
  \newcounter{contgrilla} % Para hacer ciclos
  \newcounter{columnainicial} % Se van a usar para los cline cuando un
  \newcounter{columnafinal}   % ejercicio tenga incisos.
  \newcommand{\primerafila}{}
  \newcommand{\segundafila}{}
  \newcommand{\rayitas}{} % Esto va a guardar los \cline de los
                          % ejercicios con incisos, asi queda mas bonito
  \newcommand{\anchodegrilla}{20} % Es para textpos
  \newcommand{\izquierda}{7} % Estos dos le dicen a textpos donde colocar
  \newcommand{\abajo}{2}     % la grilla
  \newcommand{\anchodecasilla}{0.4cm}
  \setcounter{columnainicial}{1}
  \setcounter{total}{0}
  \newcounter{ejercicio}
  \setcounter{ejercicio}{0}
  \renewenvironment{ejercicio}[1]
  {%
    \stepcounter{ejercicio}\textbf{\noindent Ejercicio \theejercicio. [##1
      puntos]}% Formato
    \renewcommand\@currentlabel{\theejercicio}% Esto es para las
                                % referencias
    \newcommand{\invariante}[2]{%
      {\normalfont\bfseries\ttfamily invariante}%
      \ ####1\hspace{1em}####2%
    }%
    \newcommand{\Proc}[5][result]{
      \encabezadoDeProc{####1}{####2}{####3}{####4}\hspace{1em}####5}%
  }% Aca se termina el principio del ejercicio
  {% Ahora viene el final
    % Esto suma la cantidad de incisos o 1 si no hubo ninguno
    \ifthenelse{\equal{\value{enumi}}{0}}
    {\addtocounter{total}{1}}
    {\addtocounter{total}{\value{enumi}}}
    \ifthenelse{\equal{\value{ejercicio}}{1}}{}
    {
      \g@addto@macro\primerafila{&} % Si no estoy en el primer ej.
      \g@addto@macro\segundafila{&}
    }
    \ifthenelse{\equal{\value{enumi}}{0}}
    {% No tiene incisos
      \g@addto@macro\primerafila{\multicolumn{1}{|c|}}
      \bgroup% avoid overwriting somebody else's value of \tmp@a
      \protected@edef\tmp@a{\theejercicio}% expand as far as we can
      \@xa\g@addto@macro\@xa\primerafila\@xa{\tmp@a}%
      \egroup% restore old value of \tmp@a, effect of \g@addto.. is

      \stepcounter{columnainicial}
    }
    {% Tiene incisos
      % Primero ponemos el encabezado
      \g@addto@macro\primerafila{\multicolumn}% Ahora el numero de items
      \bgroup% avoid overwriting somebody else's value of \tmp@a
      \protected@edef\tmp@a{\arabic{enumi}}% expand as far as we can
      \@xa\g@addto@macro\@xa\primerafila\@xa{\tmp@a}%
      \egroup% restore old value of \tmp@a, effect of \g@addto.. is
      % global
      % Ahora el formato
      \g@addto@macro\primerafila{{|c|}}%
      % Ahora el numero de ejercicio
      \bgroup% avoid overwriting somebody else's value of \tmp@a
      \protected@edef\tmp@a{\theejercicio}% expand as far as we can
      \@xa\g@addto@macro\@xa\primerafila\@xa{\tmp@a}%
      \egroup% restore old value of \tmp@a, effect of \g@addto.. is
      % global
      % Ahora armamos la segunda fila
      \g@addto@macro\segundafila{\multicolumn{1}{|c|}{a}}%
      \setcounter{contgrilla}{1}
      \whiledo{\value{contgrilla}<\value{enumi}}
      {%
        \stepcounter{contgrilla}
        \g@addto@macro\segundafila{&\multicolumn{1}{|c|}}
        \bgroup% avoid overwriting somebody else's value of \tmp@a
        \protected@edef\tmp@a{\alph{contgrilla}}% expand as far as we can
        \@xa\g@addto@macro\@xa\segundafila\@xa{\tmp@a}%
        \egroup% restore old value of \tmp@a, effect of \g@addto.. is
        % global
      }
      % Ahora armo las rayitas
      \setcounter{columnafinal}{\value{columnainicial}}
      \addtocounter{columnafinal}{-1}
      \addtocounter{columnafinal}{\value{enumi}}
      \bgroup% avoid overwriting somebody else's value of \tmp@a
      \protected@edef\tmp@a{\noexpand\cline{%
          \thecolumnainicial-\thecolumnafinal}}%
      \@xa\g@addto@macro\@xa\rayitas\@xa{\tmp@a}%
      \egroup% restore old value of \tmp@a, effect of \g@addto.. is
      \setcounter{columnainicial}{\value{columnafinal}}
      \stepcounter{columnainicial}
    }
    \setcounter{enumi}{0}%
    \vspace{0.2cm}%
  }%
  \newcommand{\tercerafila}{}
  \newcommand{\armartercerafila}{
    \setcounter{contgrilla}{1}
    \whiledo{\value{contgrilla}<\value{total}}
    {\stepcounter{contgrilla}\g@addto@macro\tercerafila{&}}
  }
  \newcommand{\grilla}{%
    \g@addto@macro\primerafila{&\textbf{TOTAL}}
    \g@addto@macro\segundafila{&}
    \g@addto@macro\tercerafila{&}
    \armartercerafila
    \ifthenelse{\equal{\value{total}}{\value{ejercicio}}}
    {% No hubo incisos
      \begin{textblock}{\anchodegrilla}(\izquierda,\abajo)
        \begin{tabular}{|*{\value{total}}{p{\anchodecasilla}|}c|}
          \hline
          \primerafila\\
          \hline
          \tercerafila\\
          \tercerafila\\
          \hline
        \end{tabular}
      \end{textblock}
    }
    {% Hubo incisos
      \begin{textblock}{\anchodegrilla}(\izquierda,\abajo)
        \begin{tabular}{|*{\value{total}}{p{\anchodecasilla}|}c|}
          \hline
          \primerafila\\
          \rayitas
          \segundafila\\
          \hline
          \tercerafila\\
          \tercerafila\\
          \hline
        \end{tabular}
      \end{textblock}
    }
  }%
  % \datosalumno
}

\newcommand{\datosalumno}{
  \vspace{0.4cm}
  \textbf{Apellidos:}

  \textbf{Nombres:}

  \textbf{LU:}

  \textbf{Correo electrónico:}

  \textbf{Nro. de carillas que adjunta:}
  \vspace{0.5cm}
}


% AMBIENTE CONSIGNAS
% Se usa en el TP para ir agregando las cosas que tienen que resolver
% los alumnos.
% Dentro del ambiente hay que usar \item para cada consigna

\newcounter{consigna}
\setcounter{consigna}{0}

\newenvironment{consignas}{%
  \newcommand{\consigna}{\stepcounter{consigna}\textbf{\theconsigna.}}%
  \renewcommand{\ejercicio}[1]{\item ##1 }
  \renewcommand{\proc}[5][result]{\item
    \encabezadoDeProc{##1}{##2}{##3}{##4}\hspace{1em}##5}%
  \newcommand{\invariante}[2]{\item%
    {\normalfont\bfseries\ttfamily invariante}%
    \ ##1\hspace{1em}##2%
  }
  \renewcommand{\aux}[4]{\item%
    {\normalfont\bfseries\ttfamily aux\ }%
    {\normalfont\ttfamily ##1}%
    \ifthenelse{\equal{##2}{}}{}{\ (##2)}\ : ##3 \hspace{1em}##4%
  }
  % Comienza la lista de consignas
  \begin{list}{\consigna}{%
      \setlength{\itemsep}{0.5em}%
      \setlength{\parsep}{0cm}%
    }
}%
{\end{list}}



% para decidir si usar && o ^
\newcommand{\y}[0]{\ensuremath{\land}}

% macros de correctitud
\newcommand{\semanticComment}[2]{#1 \ensuremath{#2};}
\newcommand{\namedSemanticComment}[3]{#1 #2: \ensuremath{#3};}


\newcommand{\local}[1]{\semanticComment{local}{#1}}

\newcommand{\vale}[1]{\semanticComment{vale}{#1}}
\newcommand{\valeN}[2]{\namedSemanticComment{vale}{#1}{#2}}
\newcommand{\impl}[1]{\semanticComment{implica}{#1}}
\newcommand{\implN}[2]{\namedSemanticComment{implica}{#1}{#2}}
\newcommand{\estado}[1]{\semanticComment{estado}{#1}}

\newcommand{\invarianteCN}[2]{\namedSemanticComment{invariante}{#1}{#2}}
\newcommand{\invarianteC}[1]{\semanticComment{invariante}{#1}}
\newcommand{\varianteCN}[2]{\namedSemanticComment{variante}{#1}{#2}}
\newcommand{\varianteC}[1]{\semanticComment{variante}{#1}}
\usepackage{caratula}
\usepackage{enumerate}
\usepackage{hyperref}
\usepackage{graphicx}
\usepackage{amsfonts}
\usepackage{enumitem}
\usepackage{listings} % Permite usar código

\decimalpoint
\hypersetup{colorlinks=true, linkcolor=black, urlcolor=blue}
\setlength{\parindent}{0em}
\setlength{\parskip}{0.5em}
\setcounter{tocdepth}{2} % profundidad de indice
\setcounter{section}{4} % nro de práctica - 1
\renewcommand{\thesubsubsection}{\thesubsection.\Alph{subsubsection}}
\graphicspath{ {images/} }

% End latex config

\begin{document}

\titulo{Práctica 5}
\fecha{1er cuatrimestre 2022 }
\materia{Algoritmos y Estructuras de Datos 1}
\integrante{Yago Pajariño}{546/21}{ypajarino@dc.uba.ar}

%Carátula
\maketitle
\newpage

%Indice
\tableofcontents
\newpage

% Aca empieza lo propio del documento

\section{Práctica 5}

\subsection{Ejercicio 1}

\subsubsection{Pregunta i}

\begin{itemize}
    \item $ P_c \equiv \{ i = 0; result = 0 \} $
    \item $ Q_c \equiv \{ result = \sum_{j = 0}^{|s| - 1}s[j] \} $
\end{itemize}

\subsubsection{Pregunta ii}

Falla $ \{ I \wedge B \}S\{ I \} $ en la última iteración, pues al finalizar esta, $i = |s|$

\subsubsection{Pregunta iii}

Falla $ P_c \implies I $ pues si el límite de la sumatoria es i, la misma se inicializa con $result = s[0]$.

\subsubsection{Pregunta iv}

Falla $ \{ I \wedge B \}S\{ I \} $ en la última iteración, pues se indefine $ s[i] $ en la última iteración.

\subsubsection{Pregunta v}

Defino y/o recuerdo,
\begin{itemize}
    \item $ P_c \equiv \{ i = 0 \wedge result = 0 \} $
    \item $ Q_c \equiv \{ result = \sum_{j = 0}^{|s| - 1}s[j] \} $
    \item $ I \equiv \{ 0 \leq i \leq |s| \yLuego result = \sum_{j = 0}^{i - 1}s[j] \} $
    \item $ B \equiv \{ i < |s| \} $
\end{itemize}

Para probar la correctitud del ciclo tengo que demostrar que valen:
\begin{enumerate}[label=(\alph*)]
    \item $ P_c \implies I $
    \item $ \{ I \wedge B \} S \{ I \} $
    \item $ (I \wedge \neg B) \implies Q_c $
\end{enumerate}

\textbf{Demostración (a)}

$ i = 0 \implies 0 \leq i \leq |s| $ pues $ |s| \geq 0 $

$ i = 0 \implies result = \sum_{j = 0}^{0 - 1}s[j] = 0 = result $

\textbf{Demostración (c)}

\begin{align*}
    (I \wedge \neg B) &\equiv \{ 0 \leq i \leq |s| \yLuego result = \sum_{j = 0}^{i - 1}s[j] \wedge i \geq |s| \} \\
    &\equiv \{ i = |s| \yLuego result = \sum_{j = 0}^{i - 1}s[j] \} \\
\end{align*}

Pero $ i = |s| \implies result = \sum_{j = 0}^{|s| - 1}s[j] \equiv Q_c $

\textbf{Demostración (b)}

$ \{ I \wedge B \} S \{ I \} \iff (I \wedge B) \implies wp(S, I) $

\begin{align*}
    wp(S, I) &\equiv wp(result := result + s[i], wp(i := i+1, I)) \\
    &\equiv wp(result := result + s[i], def(i+1) \yLuego (0 \leq i + 1 \leq |s| \yLuego result = \sum_{j = 0}^{i + 1 - 1}s[j])) \\
    &\equiv wp(result := result + s[i], (0 \leq i + 1 \leq |s| \yLuego result = \sum_{j = 0}^{i + 1 - 1}s[j])) \\
    &\equiv def(result + s[i]) \yLuego (0 \leq i + 1 \leq |s| \yLuego result + s[i] = \sum_{j = 0}^{i + 1 - 1}s[j]) \\
    &\equiv 0 \leq i < |s| \yLuego (0 \leq i + 1 \leq |s| \yLuego result + s[i] = \sum_{j = 0}^{i + 1 - 1}s[j]) \\
    &\equiv 0 \leq i < |s| \yLuego result + s[i] = \sum_{j = 0}^{i}s[j] \\
\end{align*}

Por $(I \wedge B)$ se que $ 0 \leq i < |s| $ y $ result + s[i] = \sum_{j = 0}^{i}s[j] \iff result = \sum_{j = 0}^{i-1}s[j] $

Luego el ciclo es parcialmente correcto.

\subsubsection{Pregunta vi}

Defino $ fv = |s| - i $

Para probar que el ciclo termina tengo que demostrar que,
\begin{enumerate}[label=(\alph*)]
    \item $ \{ I \wedge B \wedge fv = v_0 \} S \{ fv < v_0 \} $
    \item $ (I \wedge fv \leq 0) \implies \neg B $
\end{enumerate}

\textbf{Demostración (b)}

$ fv \leq 0 \iff |s| - i \leq 0 \iff |s| \leq i \equiv i \geq |s| $

$ \neg B \equiv \neg (i < |s|) \equiv i \geq |s| $

\textbf{Demostración (a)}

Tengo que probar que $ (I \wedge B \wedge fv = v_0) \implies wp(S, fv < v_0) $

\begin{align*}
    wp(S, fv < v_0) &\equiv wp(result := result + s[i], wp(i := i+1, fv < v_0)) \\
    &\equiv wp(result := result + s[i], def(i+1) \yLuego |s| - (i+1) < v_0) \\
    &\equiv wp(result := result + s[i], |s| - (i+1) < v_0) \\
    &\equiv \{def(result + s[i]) \yLuego |s| - (i+1) < v_0 \} \\
    &\equiv \{ 0 \leq i < |s| \yLuego |s| - (i+1) < v_0 \} \\
\end{align*}

Pero,
\begin{itemize}
    \item $ 0 \leq i < |s| $ vale por $ ( I \wedge B ) $
    \item $ |s| - i - 1 < v_0 \iff |s| - i - 1 < |s| - i \iff -1 < 0 $
\end{itemize}

Por lo tanto el ciclo es correcto y finaliza.

\subsection{Ejercicio 2}

Defino,
\begin{itemize}
    \item $ P_c \equiv \{ result = 0 \wedge i = 0 \} $
    \item $ Q_c \equiv \{ i = n+n \bmod 2 \wedge result = \sum_{j = 0}^{n-1}\IfThenElse{j \bmod 2 = 0}{j}{0} \} $
    \item $ I \equiv 0 \leq i \leq n+1 \wedge i \bmod 2 = 0\wedge result = \sum_{j = 0}^{i-1}\IfThenElse{j \bmod 2 = 0}{j}{0} $
    \item $ B \equiv i < n $
\end{itemize}

Para probar la correctitud del ciclo tengo que demostrar que valen:
\begin{enumerate}[label=(\alph*)]
    \item $ P_c \implies I $
    \item $ \{ I \wedge B \} S \{ I \} $
    \item $ (I \wedge \neg B) \implies Q_c $
\end{enumerate}

\textbf{Demostración (a)}

Se que $result = 0 $ y que $ i = 0 $

Quiero probar que:
\begin{itemize}
    \item $ 0 \leq i \leq n+1 $. Vale pues $n \geq 0$ e $ i = 0 $
    \item $ i \bmod 2 = 0 $. Vale pues $ 0 \bmod 2 = 0 $
    \item $ result = \sum_{j = 0}^{i-1}\IfThenElse{j \bmod 2 = 0}{j}{0} $
\end{itemize}

Con $ i = 0 \implies \sum_{j = 0}^{0-1}\IfThenElse{j \bmod 2 = 0}{j}{0} = 0= result $

Luego $ P_c \implies I $ como se quería probar.

\textbf{Demostración (c)}

\begin{align*}
    (I \wedge \neg B) \equiv \{ 0 \leq i \leq n+1 \wedge i\geq n \wedge i \bmod 2 = 0 \wedge result = \sum_{j = 0}^{i-1}\IfThenElse{j \bmod 2 = 0}{j}{0}  \}
\end{align*}

Luego se que $ n \leq i \leq n+1 \iff i =n \vee i = n+1 $

Separo en casos par e impar,
\begin{itemize}
    \item $ n$ par $\implies i = n+n \bmod 2 = n $
    \item $ n$ impar $\implies i = n+n \bmod 2 = n+1 $
\end{itemize}

Y con los valores de $i$ hallados,
\begin{itemize}
    \item $ i = n \implies result = \sum_{j = 0}^{n-1}\IfThenElse{j \bmod 2 = 0}{j}{0} $
    \item $ i = n + 1 \implies result = \sum_{j = 0}^{n}\IfThenElse{j \bmod 2 = 0}{j}{0} = \sum_{j = 0}^{n-1}\IfThenElse{j \bmod 2 = 0}{j}{0} + 0 $
\end{itemize}

Luego $ (I \wedge \neg B) \implies Q_c $ como se quería probar.

\textbf{Demostración (b)}

$ \{ I \wedge B \} S \{ I \} \iff (I \wedge B) \implies wp(S, I) $

Luego,
\begin{align*}
    wp(S, I) &\equiv wp(result := result + 1, wp(i := i+2, I)) \\
    &\equiv wp(result := result + 1, (0 \leq i+2 \leq n+1 \wedge i+2 \bmod 2 = 0\wedge result = \sum_{j = 0}^{i+2-1}\IfThenElse{j \bmod 2 = 0}{j}{0})) \\
    &\equiv def(result + 1) \yLuego (0 \leq i+2 \leq n+1 \wedge i+2 \bmod 2 = 0\wedge result + 1 = \sum_{j = 0}^{i+2-1}\IfThenElse{j \bmod 2 = 0}{j}{0}) \\
    &\equiv \{ 0 \leq i+2 \leq n+1 \wedge i \bmod 2 = 0\wedge result + i = \sum_{j = 0}^{i+1}\IfThenElse{j \bmod 2 = 0}{j}{0} \} \\
\end{align*}
Queda ver que $ (I \wedge B) \implies wp(S, I) $

Entonces, $ (I \wedge B) \equiv 0 \leq i \leq n+1 \wedge i<n \wedge i \bmod 2 = 0\wedge result = \sum_{j = 0}^{i-1}\IfThenElse{j \bmod 2 = 0}{j}{0} $

Y quiero probar,
\begin{itemize}
    \item $ 0 \leq i + 2 $. Vale pues por I, $ 0 \leq i $
    \item $ i+2 \leq n+1 $. Vale pues $ i < n \iff i+1 \leq n \iff i+2 \leq n+1 $
    \item $ i \bmod 2 = 0 $. Vale por $I$
    \item $ result = \sum_{j = 0}^{i-1}\IfThenElse{j \bmod 2 = 0}{j}{0} $
\end{itemize}

La última condicion vale pues la sumatoria entre $ 0 \leq j \leq i-1 $ vale por $I$, dado que $i$ es par, el i-ésimo termino de la sumatoria es igual a i y el $i+1$ es cero.

Por lo tanto el ciclo es parcialmente correcto respecto a su especificación.

Para probar finalización del ciclo tengo que probar:
\begin{enumerate}[label=(\alph*)]
    \item $ \{ I \wedge B \wedge f_v = v_0 \} S \{ f_v < v_0 \} $
    \item $ (I \wedge f_v \leq 0) \implies \neg B $
\end{enumerate}

Sea $ f_v = n-i $

\textbf{Demostración (b)}

$ n-i \leq 0 \iff n-i+i \leq i \iff n \leq i \equiv \neg B $

\textbf{Demostración (a)}

$ \{ I \wedge B \wedge f_v = v_0 \} S \{ f_v < v_0 \} \iff (I \wedge B \wedge f_v = v_0) \implies wp(S, f_v < v_0) $
\begin{align*}
    wp(S, f_v < v_0) &\equiv wp(result := result + 1, wp(i := i+2, f_v < v_0)) \\
    &\equiv wp(result := result + 1, n-(i+2) < v_0 ) \\
    &\equiv \{ n-(i+2) < v_0 \} \\
    &\equiv \{ n-i-2 < v_0 \} \\
\end{align*}
Pero $ f_v = v_0 \implies n-i = v_0 \implies n-i-2 = v_0 -2 < v_0 \iff -2 < 0 $

Luego el ciclo finaliza y por lo tanto es correcto respecto a su especificación.

\subsection{Ejercicio 3}

\subsubsection{Pregunta i}

\begin{lstlisting}[language = C++]
    int result = 1;
    int i = 0;
    while ( i < n ) {
        result = result * m;
        i = i + 1;
    }
\end{lstlisting}

Defino,
\begin{itemize}
    \item $ P_C \equiv \{ result = 1 \wedge i = 0 \} $
    \item $ Q_C \equiv \{ result = m^n \} $
    \item $ I \equiv \{ 0 \leq i \leq n \wedge result = m^i \} $
    \item $ B \equiv \{ i < n \} $
\end{itemize}

Para probar la correctitud del ciclo tengo que demostrar que valen:
\begin{enumerate}[label=(\alph*)]
    \item $ P_c \implies I $
    \item $ \{ I \wedge B \} S \{ I \} $
    \item $ (I \wedge \neg B) \implies Q_c $
\end{enumerate}

\textbf{Demostración (a)}

$ i = 0 \implies 0 \leq i $ y por $ Pre: n \geq 0 \implies i \leq n $

$ i = 0 \implies m^0 = 1 = result $

\textbf{Demostración (c)}

$ (I \wedge \neg B) \equiv \{ 0 \leq i \leq n \wedge result = m^i \wedge i \geq n \} $

Luego se que $ (I \wedge \neg B) \implies i = n \implies result = m^n \equiv Post $

\textbf{Demostración (b)}

$ \{ I \wedge B \} S \{ I \} \iff (I \wedge B) \implies wp(S,I) $

Luego,
\begin{align*}
    wp(S,I) &\equiv wp(result = result * m, wp(i= i+1, I)) \\
    &\equiv wp(result = result * m, def(i+1) \yLuego 0 \leq i+1 \leq n \wedge result = m^{i+1}) \\
    &\equiv wp(result = result * m,0 \leq i+1 \leq n \wedge result = m^{i+1}) \\
    &\equiv def(result * m) \yLuego 0 \leq i+1 \leq n \wedge result * m = m^{i+1} \\
    &\equiv 0 \leq i+1 \leq n \wedge result * m = m^{i+1} \\
\end{align*}

$ 0 \leq i \leq n \wedge i < n \implies 0 \leq i < n \implies   \leq i+1 \wedge i<n \implies i+1 \leq n$

$ result = m^i \iff resutl * m = m^i * m \iff result * m = m^{i+1} $

Por lo tanto queda probado que el ciclo es parcialmente correcto.

Para probar finalización del ciclo tengo que probar:
\begin{enumerate}[label=(\alph*)]
    \item $ \{ I \wedge B \wedge f_v = v_0 \} S \{ f_v < v_0 \} $
    \item $ (I \wedge f_v \leq 0) \implies \neg B $
\end{enumerate}

Sea $ f_v = n-i $

\textbf{Demostración (b)}

$ (I \wedge f_v \leq 0) \equiv \{ 0 \leq i \leq n \wedge result = m^i \wedge n-i \leq 0 \} $

Pero, $ n-i \leq 0 \implies n \leq i \implies i \geq n \equiv \neg B $

\textbf{Demostración (a)}

$ \{ I \wedge B \wedge f_v = v_0 \} S \{ f_v < v_0 \} \iff (I \wedge B \wedge f_v = v_0) \implies wp(S, f_v < v_0) $ 
\begin{align*}
    wp(S,f_v < v_0) &\equiv wp(result = result * m, wp(i= i+1, n-i < v_0)) \\
    &\equiv wp(result = result * m, def(i+1) \yLuego n-(i+1) < v_0) \\
    &\equiv wp(result = result * m, n-(i+1) < v_0) \\
    &\equiv \{ def(result * m) \yLuego n-(i+1) < v_0 \} \\
    &\equiv \{ n-(i+1) < v_0 \} \\
\end{align*}

Pero, $ n-i = v_0 \implies n-i-1 < v_0 \iff v_0 - 1 < v_0 \iff -1 < 0 $

Por lo tanto el ciclo finaliza. Y el programa es correcto respecto a su especificación.

\subsubsection{Pregunta ii}

Falla la demostración de $ \{ I \wedge B \}S\{ I \} $

\subsubsection{Pregunta iii}

Es correcto, solo hay que probar de nuevo los puntos (2) y (4) de la demostración.

\subsubsection{Pregunta iv}

Se puede pedir $ n \geq 2 $ en la precondición.

\subsection{Ejercicio 4}

\subsubsection{Pregunta i}

\begin{lstlisting}[language = C++]
    int i = 1;
    int result = 0;
    while (i <= n) {
        if (n % i == 0) {
            result = result + 1;
        }
        i = i + 1;
    }
\end{lstlisting}

\subsubsection{Pregunta ii}

El invariante propuesto falla en la última iteración, hay que cambiarlo por \\
$ \{ 1 \leq i \leq n \wedge result = \sum_{j = 1}^{i}\IfThenElse{n \bmod j = 1}{j}{0} \} $

\subsection{Ejercicio 5}

\subsubsection{Pregunta i}
\pagebreak
\begin{lstlisting}[language = C++]
    int result = 0;
    int j = 0;
    while (j < s.size()) {
        if (j % 2 == 1) {
            result = result + s[j];
        }
        j = j+1;
    }
\end{lstlisting}

\subsubsection{Pregunta ii}

Defino,
\begin{itemize}
    \item $ P_C \equiv \{ result = 0 \wedge j = 0 \} $
    \item $ Q_C \equiv \{ result = \sum_{i=0}^{|s|-1}\IfThenElse{i \bmod 2 = 1}{s[i]}{0} \} $
    \item $ I \equiv \{ 0 \leq j \leq |s| \wedge result = \sum_{i=0}^{j-1}\IfThenElse{i \bmod 2 = 1}{s[i]}{0} \} $
    \item $ B \equiv \{ j < |s| \} $
    \item $ f_v = |s| - j $
\end{itemize}

Para probar la correctitud del ciclo tengo que demostrar que valen:
\begin{enumerate}[label=(\alph*)]
    \item $ P_c \implies I $
    \item $ \{ I \wedge B \} S \{ I \} $
    \item $ (I \wedge \neg B) \implies Q_c $
    \item $ \{ I \wedge B \wedge f_v = v_0 \} S \{ f_v < v_0 \} $
    \item $ (I \wedge f_v \leq 0) \implies \neg B $
\end{enumerate}

\textbf{Demostración (a)}

$ j = 0 \implies 0 \leq j \leq |s| $

$ j = 0 \implies \sum_{i=0}^{j-1}\IfThenElse{i \bmod 2 = 1}{s[i]}{0} = 0 = result $

\textbf{Demostración (c)}

$ 0 \leq j \leq |s| \wedge j \geq |s| \implies j = |s| \implies result = \sum_{i=0}^{|s|-1}\IfThenElse{i \bmod 2 = 1}{s[i]}{0} \equiv Q_c $

\textbf{Demostración (b)}

$ \{ I \wedge B \} S \{ I \} \iff (I \wedge B) \implies wp(S, I) $

Luego,
\begin{align*}
    wp(S, I) &\equiv wp(if(...), wp(j = j+1, I)) \\
    &\equiv wp(if(...), def(j+1) \yLuego (0 \leq j+1 \leq |s| \wedge result = \sum_{i=0}^{j+1-1}\IfThenElse{i \bmod 2 = 1}{s[i]}{0})) \\
    &\equiv wp(if(...), (0 \leq j+1 \leq |s| \wedge result = \sum_{i=0}^{j}\IfThenElse{i \bmod 2 = 1}{s[i]}{0})) \\
    &\equiv (j \bmod 2 = 1 \wedge wp(result = result +1,(0 \leq j+1 \leq |s| \wedge result = \sum_{i=0}^{j}\IfThenElse{i \bmod 2 = 1}{s[i]}{0}))) \vee \\
    &\quad \; (j \bmod 2 = 0 \wedge wp(skip,(0 \leq j+1 \leq |s| \wedge result = \sum_{i=0}^{j}\IfThenElse{i \bmod 2 = 1}{s[i]}{0}))) \\
    &\equiv (j \bmod 2 = 1 \wedge (0 \leq j+1 \leq |s| \wedge result+1 = \sum_{i=0}^{j}\IfThenElse{i \bmod 2 = 1}{s[i]}{0})) \vee \\
    &\quad \; (j \bmod 2 = 0 \wedge (0 \leq j+1 \leq |s| \wedge result = \sum_{i=0}^{j}\IfThenElse{i \bmod 2 = 1}{s[i]}{0})) \\
\end{align*}
Por $ (I \wedge B) $ vale que $ 0 \leq j+1 \leq |s| $ en ambas ramas del if.

Ambas sumatorias valen pues dependiendo de la paridad de $j$, se suma $1$ a la sumatoria. Y la sumatoria entre $ 0 \leq j \leq |s| $ vale por $I$

\textbf{Demostración (e)}

$ |s| - j < 0 \implies j \geq |s| \equiv \neq B $

\textbf{Demostración (d)}

$ \{ I \wedge B \wedge f_v = v_0 \} S \{ f_v < v_0 \} \iff (I \wedge B \wedge f_v = v_0) \implies wp(S, f_v < v_0) $
\begin{align*}
    wp(S, f_v < v_0) &\equiv wp(if(...), wp(j = j+1, f_v < v_0)) \\
    &\equiv wp(if(...), def(j+1) \yLuego |s| - (j+1) < v_0) \\
    &\equiv wp(if(...), |s| - (j+1) < v_0) \\
    &\equiv (j \bmod 2 = 1 \wedge |s| - (j+1) < v_0) \vee ((j \bmod 2 \neq 1 \wedge |s| - (j+1) < v_0)) \\
    &\equiv \{ |s| - (j+1) < v_0 \} \\
\end{align*}

Luego $ |s| - j - 1 < v_0 \iff v_0 -1 < v_0 \iff -1 < 0 $

Por lo tanto el ciclo es correcto.

\subsection{Ejercicio 6}

\subsubsection{Pregunta i}

\begin{itemize}
    \item $ P_c \equiv \{ i = 0 \wedge j = 1 \} $
    \item $ Q_c \equiv Post $
\end{itemize}

\subsubsection{Pregunta ii}

Defino,
\begin{itemize}
    \item $ P_C \equiv \{ i = 0 \wedge j = 1 \} $
    \item $ Q_C \equiv Post $
    \item $ I \equiv \{ (0 \leq i < |s| \wedge 1 \leq j \leq |s|) \yLuego (\forall k: \ent)(0 \leq k < j \implicaLuego s[k] \leq s[i]) \} $
    \item $ B \equiv \{ j < |s| \} $
\end{itemize}

Para probar la correctitud parcial del ciclo tengo que demostrar que valen:
\begin{enumerate}[label=(\alph*)]
    \item $ P_c \implies I $
    \item $ \{ I \wedge B \} S \{ I \} $
    \item $ (I \wedge \neg B) \implies Q_c $
\end{enumerate}

\textbf{Demostración (a)}

$ i = 0 \wedge j = 1 \implies 0 \leq 0 \leq |s| \wedge 1 \leq 1 \leq |s| \wedge k = 0 \implies s[0] = s[0] $

\textbf{Demostración (c)}

$ 1 \leq j \leq |s| \wedge j \geq |s| \implies j = |s| \implies (\forall k: \ent)(0 \leq k \leq |s| \implicaLuego s[k] \leq s[i]) \equiv Q_C $

\textbf{Demostración (b)}

$ \{ I \wedge B \} S \{ I \} \iff (I \wedge B) \implies wp(C, I) $
\begin{align*}
    wp(C, I) &\equiv wp(iff(...), wp(j=j+1, I)) \\
    &\equiv wp(if(...), def(j+1) \yLuego (0 \leq i < |s| \wedge 1 \leq j+1 \leq |s|) \yLuego (\forall k: \ent)(0 \leq k < j+1 \implicaLuego s[k] \leq s[i])) \\
    &\equiv wp(if(...), (0 \leq i < |s| \wedge 1 \leq j+1 \leq |s|) \yLuego (\forall k: \ent)(0 \leq k < j+1 \implicaLuego s[k] \leq s[i])) \\
    &\equiv def(s[j] > s[i]) \yLuego (s[j] > s[i] \wedge wp(i = j, (0 \leq i < |s| \wedge 1 \leq j+1 \leq |s|) \yLuego (\forall k: \ent)(0 \leq k < j+1 \implicaLuego s[k] \leq s[i])) \vee \\
    &\qquad \qquad \qquad \qquad \quad \; \; (s[j] \leq s[i] \wedge wp(skip, (0 \leq i < |s| \wedge 1 \leq j+1 \leq |s|) \yLuego (\forall k: \ent)(0 \leq k < j+1 \implicaLuego s[k] \leq s[i])))) \\
    &\equiv 0 \leq j < |s| \wedge 0 \leq i < |s| \yLuego (s[j] > s[i] \wedge (0 \leq j < |s| \wedge 1 \leq j+1 \leq |s|) \yLuego (\forall k: \ent)(0 \leq k < j+1 \implicaLuego s[k] \leq s[j])) \vee \\
    &\qquad \qquad \qquad \qquad \qquad \qquad \quad \; (s[j] \leq s[i] \wedge (0 \leq i < |s| \wedge 1 \leq j+1 \leq |s|) \yLuego (\forall k: \ent)(0 \leq k < j+1 \implicaLuego s[k] \leq s[i])) \\
\end{align*}

\subsubsection{Pregunta ii}

$ f_v = |s| - j -1 $

TODO la demostración de finalización.

\subsection{Ejercicio 7}

\subsubsection{Pregunta i}

\begin{itemize}
    \item $ P_c \equiv i = 0 \wedge |r| = |s| $
    \item $ Q_c \equiv Post $
\end{itemize}

\subsubsection{Pregunta ii}

Defino,
\begin{itemize}
    \item $ P_C \equiv \{ i = 0 \wedge |r| = |s| \} $
    \item $ Q_C \equiv Post $
    \item $ I \equiv \{ 0 \leq i \leq |s| \wedge |s| = |r| \wedge (\forall j: \ent)(0 \leq j < i \implicaLuego s[j] = r[j]) \} $
    \item $ B \equiv \{ i < |s| \} $
\end{itemize}

Para probar la correctitud parcial del ciclo tengo que demostrar que valen:
\begin{enumerate}[label=(\alph*)]
    \item $ P_c \implies I $
    \item $ \{ I \wedge B \} S \{ I \} $
    \item $ (I \wedge \neg B) \implies Q_c $
\end{enumerate}

\textbf{Demostración (a)}

$ i = 0 \implies 0 \leq i \leq |s| $

$ |s| = |r| \implies |s| = |r| $

$ i = 0 \implies (\forall j: \ent)(0 \leq j < 0 \implicaLuego s[j] = r[j]) $ pero $ 0 \leq j < 0 \equiv False $

\textbf{Demostración (c)}

$ (I \wedge \neg B) \equiv 0 \leq i \leq |s| \wedge |s| = |r| \wedge (\forall j: \ent)(0 \leq j < i \implicaLuego s[j] = r[j]) \wedge i \geq |s| $

Pero, $ 0 \leq i \leq |s| \wedge i \geq |s| \implies i = |s| $

Luego, $ i = 0 \implies |s| = |r| \wedge (\forall j: \ent)(0 \leq j < |s| \implicaLuego s[j] = r[j]) \equiv Q_c $

\textbf{Demostración (b)}

$ \{ I \wedge B \} S \{ I \} \iff (I \wedge B) \implies wp(S,I) $

\begin{align*}
    wp(S,I) &\equiv wp(r[i] = s[i], wp(i=i+1, I)) \\
    &\equiv wp(r[i] = s[i], def(i+1) \yLuego 0 \leq i+1 \leq |s| \wedge |s| = |r| \wedge (\forall j: \ent)(0 \leq j < i+1 \implicaLuego s[j] = r[j])) \\
    &\equiv wp(r = setAt(r,i,s[i]), 0 \leq i+1 \leq |s| \wedge |s| = |r| \wedge (\forall j: \ent)(0 \leq j < i+1 \implicaLuego s[j] = r[j])) \\
    &\equiv 0 \leq i < |s| \yLuego |s| = |r| \wedge (\forall j: \ent)(0 \leq j < i \implicaLuego s[j] = r[j])) \\
\end{align*}
Es facil ver que por $ I\wedge B $ valen los tres elementos de la conjunción.

Luego el ciclo es parcialmente correcto.

\subsubsection{Pregunta iii}

Defino $ f_v = |s| - i $

Para probar finalización del ciclo tengo que probar:
\begin{enumerate}[label=(\alph*)]
    \item $ \{ I \wedge B \wedge f_v = v_0 \} S \{ f_v < v_0 \} $
    \item $ (I \wedge f_v \leq 0) \implies \neg B $
\end{enumerate}

\textbf{Demostración (b)}

$ (I \wedge f_v \leq 0) \implies |s| - i \leq 0 \implies i \geq |s| \equiv \neg B $

\textbf{Demostración (a)}

$ \{ I \wedge B \wedge f_v = v_0 \} S \{ f_v < v_0 \} \iff (I \wedge B \wedge f_v = v_0) \implies wp(S, f_v < v_0) $
\begin{align*}
    wp(S,I) &\equiv wp(r[i] = s[i], wp(i=i+1, |s| - i < v_0)) \\
    &\equiv wp(r[i] = s[i], |s| - (i+1) < v_0) \\
    &\equiv |s| - (i+1) < v_0 \\
\end{align*}

Pero $ f_v = v_0 \implies v_0 - 1 < v_0 \iff -1 < 0 $

Así queda probado que el ciclo finaliza.

\subsection{Ejercicio 8}

\subsubsection{Pregunta i}

\begin{itemize}
    \item $ P_c \equiv \{ i = d \} $
    \item $ q_c \equiv Post $
\end{itemize}

\subsubsection{Pregunta ii}

Defino,
\begin{itemize}
    \item $ P_C \equiv \{ i = d \} $
    \item $ Q_C \equiv Post $
    \item $ I \equiv \{ d \leq i \leq |s| \wedge |s| = |s_0| \yLuego (\forall j: \ent)(0 \leq j < d \implicaLuego s[j] = s_0[j]) \wedge (\forall j: \ent)(d \leq j < |s| \implicaLuego s[j] = e) \} $
    \item $ B \equiv \{ i < |s| \} $
\end{itemize}

Para probar la correctitud parcial del ciclo tengo que demostrar que valen:
\begin{enumerate}[label=(\alph*)]
    \item $ P_c \implies I $
    \item $ \{ I \wedge B \} S \{ I \} $
    \item $ (I \wedge \neg B) \implies Q_c $
\end{enumerate}

TODO. Hecho en clase 20/04/2022.

\subsection{Ejercicio 9}

\subsubsection{Pregunta i}

Defino,
\begin{itemize}
    \item $ P_C \equiv \{ |s| \bmod 2 = 0 \wedge i = |s| - 1\wedge suma = 0 \} $
    \item $ Q_C \equiv \{ |s| \bmod 2 = 0 \wedge i = |s| / 2 -1 \yLuego suma = \sum_{j=0}^{|s|/2-1}s[j] \} $
    \item $ I \equiv \{ |s|/2 - 1 \leq i \leq |s|-1 \wedge |s| \bmod 2 = 0 \wedge suma = \sum_{j=0}^{|s|-2-i} \} $
    \item $ B \equiv \{ i \geq |s|/2 \} $
\end{itemize}

Para probar la correctitud parcial del ciclo tengo que demostrar que valen:
\begin{enumerate}[label=(\alph*)]
    \item $ P_c \implies I $
    \item $ \{ I \wedge B \} S \{ I \} $
    \item $ (I \wedge \neg B) \implies Q_c $
\end{enumerate}

\subsubsection{Pregunta ii}

$ f_v = i - |s|/2 - 1 $

\subsubsection{Pregunta iii}

TODO

\subsection{Ejercicio 10}

\subsubsection{Pregunta i}

\begin{lstlisting}[language = C++]
    int i = 0;
    while (i < s.size()) {
        if (s[i] == a) {
            s[i] = b;
        }
        i = i+1;
    }
\end{lstlisting}

\subsubsection{Pregunta ii}

\begin{itemize}
    \item $ P_c \equiv i = 0 \wedge |s| = |s_0| $
    \item $ Q_c \equiv Post $
\end{itemize}

\subsubsection{Pregunta iii}

Defino,
\begin{itemize}
    \item $ P_C \equiv \{ i = 0 \wedge |s| = |s_0| \} $
    \item $ Q_C \equiv Post $
    \item $ I \equiv \{ |s| = |s_0| \wedge 0 \leq i \leq |s| \wedge (\forall j: \ent)((0 \leq j < i \yLuego s_0[j] = a) \implicaLuego s[j] = b) \wedge (\forall j: \ent)((0 \leq j < i \yLuego s_0[j] \neq a) \implicaLuego s[j] = s_0[j]) \} $
    \item $ B \equiv \{ i < |s| \} $
\end{itemize}

Para probar la correctitud parcial del ciclo tengo que demostrar que valen:
\begin{enumerate}[label=(\alph*)]
    \item $ P_c \implies I $
    \item $ \{ I \wedge B \} S \{ I \} $
    \item $ (I \wedge \neg B) \implies Q_c $
\end{enumerate}

\textbf{Demostración (a)}

$ i = 0 \implies 0 \leq i \leq |s| $

Ambos $ \forall $ son verdaderos pues $ i = 0 $ implica antecedente falso y por lo tanto implicación verdadera.

\textbf{Demostración (c)}

$ (I \wedge \neg B) \implies 0 \leq i \leq |s| \wedge i \geq |s| \implies i = |s| \implies $ ambos $ \forall $ igual que en $Post$

\textbf{Demostración (b)}

$ \{ I \wedge B \} S \{ I \} \iff (I \wedge B) \implies wp(S, I) $
\begin{align*}
    wp(S, I) &\equiv wp((if(...), wp(i=i+1, I))) \\
    &\equiv wp((if(...), I(i+1))) \\
    &\equiv def(i < |s|) \yLuego (i < |s| \wedge wp(s=setAt(s, i, b), I(i+1))) \vee (i \geq |s| \wedge wp(skip, I(i+1))) \\
    &\equiv (i < |s| \wedge 0 \leq i < |s| \yLuego |s| = |s_0| \wedge 0 \leq i+1 \leq |s| \wedge (\forall j: \ent)((0 \leq j < i+1 \yLuego s_0[j] = a) \implicaLuego setAt(s, i, b)[j] = b) \wedge \\
    &\qquad \qquad \qquad \qquad \qquad \qquad \qquad \qquad \qquad \qquad \qquad \qquad (\forall j: \ent)((0 \leq j < i+1 \yLuego s_0[j] \neq a) \implicaLuego setAt(s, i, b)[j] = s_0[j])) \vee \\
    &\quad (i \geq |s| \wedge |s| = |s_0| \wedge 0 \leq i+1 \leq |s| \wedge (\forall j: \ent)((0 \leq j < i+1 \yLuego s_0[j] = a) \implicaLuego s[j] = b) \wedge \\
    &\qquad \qquad \qquad \qquad \qquad \qquad \qquad \qquad \quad \; \; (\forall j: \ent)((0 \leq j < i+1 \yLuego s_0[j] \neq a) \implicaLuego s[j] = s_0[j])) \\
\end{align*}
Y usando $ I \wedge B $ hay que probar cada termino de la conjunción.

\subsubsection{Pregunta iv}

$ f_v = |s| - i $

\subsection{Ejercicio 11}

Tengo que probar que vale la tripla de Hoare $ \{Pre\} S \{Post\} $

La estrategia en la demostración en programas completos que tienen un ciclo es dividir en tres:
\begin{enumerate}
    \item $ Pre \implies wp(A, P_c) $
    \item $ \{P_c\} C \{Q_c\} $
    \item $ Q_c \implies wp(B, Q_c) $
\end{enumerate}
Con,
\begin{itemize}
    \item A: código antes del ciclo
    \item C: ciclo
    \item B: código después del ciclo
\end{itemize}
Por lo tanto tengo que probar,
\begin{enumerate}[label=(\alph*)]
    \item $ wp(S_4, Post) $
    \item $ wp(C, (a)) $
    \item $ wp(S_2, (b)) $
    \item $ wp(S_1, (c)) $
\end{enumerate}

\textbf{Demostración (a)}
\begin{align*}
    wp(S_4, Post) &\equiv def(j) \yLuego j = -1 \implies (\forall j: \ent)(0 \leq j < |s| \implicaLuego s[j] \neq e) \wedge \\
    &\qquad \qquad \quad \; \; \; j \neq -1 \implies (0 \leq j < |s| \yLuego s[j] = e)
\end{align*}

Esto es lo que uso como $Pre$ para probar el ciclo usando el teorema del invariante.

\textbf{Demostración (b)}

Defino,
\begin{itemize}
    \item $ P_c \equiv i = |s| - 1 \wedge j = -1 $
    \item $ Q_c \equiv (a) $
    \item $ I \equiv -1 \leq i \leq |s|-1 \wedge j = -1 \implies (\forall j: \ent)(i \leq j < |s| \implicaLuego s[j] \neq e) \wedge
    j \neq -1 \implies (i < j < |s| \yLuego s[j] = e) $
    \item $ B \equiv i \geq 0 $
\end{itemize}
Para probar la correctitud parcial del ciclo tengo que demostrar que valen:
\begin{enumerate}[label=(\alph*)]
    \item $ P_c \implies I $
    \item $ \{ I \wedge B \} S \{ I \} $
    \item $ (I \wedge \neg B) \implies Q_c $
\end{enumerate}
\textbf{Demostración (b.a)}

Se que $ i = |s|-1 \wedge j = -1 $. Luego valen,
\begin{itemize}
    \item $ -1 \leq i \leq |s|-1 $
    \item $ j = -1 \implies  (\forall j: \ent)(i \leq j < |s| \implicaLuego s[j] \neq e) $
    \item $ j \neq -1 \implies (...) $ pues el antecedente es falso.
\end{itemize}

\textbf{Demostración (b.c)}

Por $ (I \wedge \neg B) $ se que $ i = -1 $ luego valen ambos casos $ j = -1 \wedge j \neq -1 $

\textbf{Demostración (b.b)}

$ \{ I \wedge B \} S \{ I \} \iff (I \wedge B) \implies wp(S, I) $
\begin{align*}
    wp(S, I) &\equiv wp(if(...), wp(i=i-1, I)) \\
    &\equiv wp(if(...), def(i-1) \yLuego I(i-1)) \\
    &\equiv def(s[i]=e) \yLuego (s[i]=e \wedge wp(j=i, I(i-1))) \vee (s[i]\neq e \wedge wp(skip, I(i-1))) \\
    &\equiv 0 \leq i < |s| \yLuego (s[i]=e \wedge I(i = i-1)(j = i)) \vee (s[i]\neq e \wedge I(i-1)) \\
    &\equiv 0 \leq i < |s| \yLuego (s[i]=e \wedge -1 \leq i-1 \leq |s|-1 \wedge \\ 
    &i = -1 \implies (\forall j: \ent)(i-1 \leq j < |s| \implicaLuego s[j] \neq e) \wedge \\ 
    &i \neq -1 \implies (i-1 < i < |s| \yLuego s[i] = e)) \vee \\
    &(s[i]\neq e \wedge -1 \leq i-1 \leq |s|-1 \wedge \\
    &j = -1 \implies (\forall j: \ent)(i-1 \leq j < |s| \implicaLuego s[j] \neq e) \wedge \\
    &j \neq -1 \implies (i-1 < j < |s| \yLuego s[j] = e)) \\
\end{align*}

Queda probar que $ (I \wedge B) \implies wp(S,I) $ calculada

Ahora pruebo terminación del ciclo

Defino $ f_v = i+1 $

Para probar finalización del ciclo tengo que probar:
\begin{enumerate}[label=(\alph*)]
    \item $ \{ I \wedge B \wedge f_v = v_0 \} S \{ f_v < v_0 \} $
    \item $ (I \wedge f_v \leq 0) \implies \neg B $
\end{enumerate}

Luego, $ \{ I \wedge B \wedge f_v = v_0 \} S \{ f_v < v_0 \} \iff (I \wedge B \wedge f_v = v_0) \implies wp(S, f_v < v_0) $
\begin{align*}
    wp(S, f_v < v_0) &\equiv wp(if(...), wp(i=i-1, f_v < v_0)) \\
    &\equiv wp(if(...), i-1+1 < v_0) \\
    &\equiv def(s[i]=e) \yLuego (s[i]=e \wedge wp(j=i, i-1+1 < v_0)) \vee (s[i] \neq e \wedge i-1+1 < v_0) \\
    &\equiv 0 \leq i < |s| \yLuego i-1+1 < v_0 \\
\end{align*}
Luego por $ f_v = v_0 \implies i+1 = v_0 \implies i-1+1 < v_0 \equiv v_0 -1 < v_0 \iff -1 < 0 $

Además, $ f_v \leq 0 \implies i+1 \leq 0 \implies i \leq -1 \iff i < 0 \equiv \neg B $

Luego el ciclo es correcto y finaliza. Por monotonía queda probar que $ Pre \implies wp(S_1, S_2; P_c) $
\begin{align*}
    wp(S_1, S_2; P_c) &\equiv wp(S_1, wp(S_2, P_c)) \\
    &\equiv wp(i = |s|-1, wp(j=-1, P_c)) \\
    &\equiv wp(i = |s|-1, i = |s| -1 \wedge -1=-1) \\
    &\equiv wp(i = |s|-1, i = |s| -1) \\
    &\equiv def(|s|-1) \yLuego |s| -1 = |s| -1 \\
    &\equiv True \\
\end{align*}
Queda ver que $ Pre \implies True \iff True \implies True $ que es verdadero y por lo tanto el programa es correcto respecto a su especificación.

\subsection{Ejercicio 12}

Divido la demostración en:
\begin{enumerate}[label=(\alph*)]
    \item Codigo antes del ciclo $ Pre \implies wp(A, P_c) $
    \item Demostración de correctitud del ciclo usando el teorema del invariante
    \item Código posterior al ciclo $ Q_c \implies wp(C, Post) $
\end{enumerate}

\textbf{Demostración (c)}

Calculo,
\begin{align*}
    wp(if(...), Post) &\equiv def(j \neq -1) \yLuego (j \neq -1 \wedge wp(r = True, Post)) \vee \\
    &\qquad \qquad \qquad \qquad \; \:( j = -1 \wedge wp(r = false, Post)) \\
    &\equiv (j \neq -1 \wedge wp(r = True, Post)) \vee \\
    &\quad \;( j = -1 \wedge wp(r = false, Post)) \\
    &\equiv (j \neq -1 \wedge true = true \iff ((\exists k: \ent)(0 \leq k < |s| \yLuego s[k]=e))) \vee \\
    &\quad \;( j = -1 \wedge false = true \iff ((\exists k: \ent)(0 \leq k < |s| \yLuego s[k]\neq e)))
\end{align*}

\textbf{Demostración (b)}

Defino,
\begin{itemize}
    \item $ P_c \equiv i = 0 \wedge j = -1 $
    \item $ Q_c \equiv (c) $
    \item $ I \equiv 0 \leq i \leq |s| \wedge (j \neq -1 \implies (\exists k: \ent)(0 \leq k < i \yLuego s[k]=e)) \wedge (j = -1 \implies (\exists k: \ent)(0 \leq k < i \yLuego s[k]\neq e)) $
    \item $ B \equiv i \geq |s| $
\end{itemize}
Para probar la correctitud parcial del ciclo tengo que demostrar que valen:
\begin{enumerate}[label=(\alph*)]
    \item $ P_c \implies I $
    \item $ \{ I \wedge B \} S \{ I \} $
    \item $ (I \wedge \neg B) \implies Q_c $
\end{enumerate}

\textbf{Demostración (b.a)}

Se que $ i = 0 \wedge j = -1 $ luego $ 0 \leq 0 \leq |s| $ es verdadero, la implicación de $ j \neq -1 $ es verdadera, la implicación de $ j = -1 $ es verdadera pues el antecedente del $ \exists $ es falso.

\textbf{Demostración (b.c)}

$ (I \wedge \neg B) \implies i = |s| $ y por lo tanto para este valor de i, las dos ramas del invariante son iguales a las de $ Q_c $

\textbf{Demostración (b.b)}

$ \{ I \wedge B \} S \{ I \} \iff (I \wedge B) \implies wp(S, I) $

TODO

\subsection{Ejercicio 13}
TODO

\subsection{Ejercicio 14}
TODO

\subsection{Ejercicio 15}
TODO

\end{document}